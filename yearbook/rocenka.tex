\documentclass[11pt,oneside]{article}

% proper czech language
\usepackage[utf8]{inputenc}
\usepackage[T1]{fontenc}
\usepackage[czech]{babel}

\usepackage{pdfpages}
\usepackage{fancyhdr}
\usepackage[hidelinks]{hyperref}
\usepackage{titlesec}
\usepackage{booktabs}

\usepackage{xargs}
\usepackage[bottom=2cm,left=2cm,right=2cm,top=2cm]{geometry}

% to preserve links into pdfpages
\usepackage{pax}
\usepackage{xcolor}

\titleformat*{\section}{\LARGE\bfseries}

% hide horizontal lines, cause they are already in the articles
\renewcommand{\headrulewidth}{0pt}
\renewcommand{\footrulewidth}{0pt}

\newcommand{\dayHeader}[1]{
    \noindent\Large{{\color{blue}\textbf{#1}}}
}

\newcommand{\clanek}[3]{
    \includepdf[
        pages=-,
        pagecommand={},
        addtotoc={
            1,subsection,2,
            #3 (\textit{#2}),
            #1}
    ]
    {./papers/#1.pdf}
}

% sponsors' footer for title page
\newcommandx{\sponzori}[5]{%
       \textit{Za podpory}
       \vspace{0.1cm}
       \hrule
\begin{figure}[b]
         % \href{\urlSponsor1}{\includegraphics[height=45px]{img/sponzori/#1}}
         % \qquad
         % \href{\urlSponsor2}{\includegraphics[height=45px]{img/sponzori/#2}}
         % \qquad
         % \href{\urlSponsor3}{\includegraphics[height=45px]{img/sponzori/#3}}
         % \qquad
         % \href{\urlSponsor4}{\includegraphics[height=45px]{img/sponzori/#4}}
         % \qquad
         % \href{\urlSponsor5}{\includegraphics[height=45px]{img/sponzori/#5}}
\end{figure}
}

\fancypagestyle{patickaSponzoru}
{
    \fancyhead{}
    \fancyfoot[C]{
        \href{\urlSponsor1}{\includegraphics[height=35px]{./img/sponzori/logo1}}
        \qquad
        \href{\urlSponsor2}{\includegraphics[height=35px]{./img/sponzori/logo2}}
        \qquad
        \href{\urlSponsor3}{\includegraphics[height=35px]{./img/sponzori/logo3}}
        \qquad
        \href{\urlSponsor4}{\includegraphics[height=35px]{./img/sponzori/logo4}}
        \qquad
        \href{\urlSponsor5}{\includegraphics[height=35px]{./img/sponzori/logo5}}
    }
}

\fancypagestyle{paper}
{
    \fancyhead{}
    \fancyfoot[C]{\thepage}
}

\newcommand{\event}[4]{
    \small{#1} & \small{--} & \small{#2} & \small{#4} \\
     &  &  & (\small{\textit{#3}}) \\
}
\newcommand{\eventNoPer}[3]{
    \small{#1} & \small{--} & \small{#2} & \small{#3} \\
}

\newcommand{\bigEvent}[2]{
    \multicolumn{3}{l}{\color{red}\textbf{#1}} & \textbf{\color{red}#2} \\
}

\newcommand{\mezera}{
    \rule{0pt}{1ex} & & & \\
}

\newcommand{\oceneny}[3]{
    #1~místo & \textit{#2:} #3 \\
}

\title{to mikulas}
\author{rzehumat}

\date{April 2021}

\pagestyle{fancy}
\fancyhead{}
\fancyfoot[C]{\thepage}

\newcommand{\comma}{,}
\input{items.tex}

\begin{document}

\begin{titlepage}
   \begin{center}
       \vspace*{4cm}

        \LARGE{\textbf{Jaderná energetika v~pracích mladé generace~--~\thisYear}}

        \vspace{0.8cm}

        \Large{Mikulášské setkání Mladé generace ČNS}

        \vfill

        PLACE\\
        DATE\\

        \vfill

        \sponzori{sponsor1}{sponsor2}{sponsor3}{sponsor4}{sponsor5}

    \end{center}
\end{titlepage}


\tableofcontents
\newpage

\setcounter{secnumdepth}{-1}

\pagestyle{patickaSponzoru}
\section{Program setkání}
\dayHeader{Středa DATE}
\begin{table}[h]
    \begin{tabular}{r c l p{0.85\textwidth}}
        \bigEvent{18:00}{Společná procházka Brnem a výstup na věž Staré radnice}
        \bigEvent{19:00}{Večerní posezení}
    \end{tabular}
\end{table}

\dayHeader{Čtvrtek DATE}
\begin{table}[h]
    \begin{tabular}{r c l p{0.85\textwidth}}
        \bigEvent{9:00}{Oficiální zahájení Mikulášského setkání}
        \bigEvent{9:10}{Předání cen a prezentace oceněných diplomantů}
        \eventNoPer{9:10}{9:20}{Představení soutěže závěrečných prací}
        \eventNoPer{9:20}{9:40}{Oceněný BP -- 3. místo}
        \eventNoPer{9:40}{10:00}{Oceněný BP -- 2. místo}
        \eventNoPer{10:00}{10:20}{Oceněný BP -- 1. místo}
        \eventNoPer{10:20}{10:30}{Přestávka}
        \eventNoPer{10:30}{10:50}{Oceněný DP -- 3. místo}
        \eventNoPer{10:50}{11:10}{Oceněný DP -- 2. místo}
        \eventNoPer{11:10}{11:30}{Oceněný DP -- 1. místo}
        \event{11:30}{11:45}{}{Představení Mladé generace České nukleární společnosti}
        \eventNoPer{11:45}{12:00}{Přestávka}
        \bigEvent{12:00}{Prezentace hosta setkání}
        \event{}{}{}{}
        \bigEvent{13:00}{Oběd}
        \bigEvent{14:00}{Prezentace prací mladých odborníků}
        \event{14:00}{14:20}{}{}
        \event{14:20}{14:40}{}{}
        \event{14:40}{15:00}{}{}
        \event{15:00}{15:20}{}{}
        \event{15:20}{15:40}{}{}
        \event{15:40}{16:00}{}{}
        \bigEvent{16:00}{Zakončení setkání}
        \bigEvent{17:30}{Večerní posezení}
    \end{tabular}
\end{table}

\dayHeader{Pátek DATE}
\begin{table}[h]
    \begin{tabular}{r c l p{0.85\textwidth}}
        \bigEvent{9:00}{Exkurze}
    \end{tabular}
\end{table}
\newpage

\pagestyle{patickaSponzoru}

\section{O Mikulášském setkání}
 {\normalsize Mikulášské setkání je akcí Mladé generace České nukleární společnosti určené především mladým lidem studujícím či pracujícím v~jaderné oblasti. Smyslem setkání je propojit mladé lidi z~různých koutů České republiky a~umožnit jim prezentovat jejich práci. Hlavní náplní Mikulášského setkání je prezentace oceněných bakalářských a~magisterských prací v~soutěži, která je pravidelně pořádána Českou nukleární společností společně s~ÚJV Řež.}

\vspace{1cm}

\section{O Mladé generaci ČNS}
 {\normalsize Mladá generace České nukleární společnosti (CYG - Czech Young Generation) byla založena 7.~7.~1997. CYG je jednou ze sekcí České nukleární společnsti a~je také členem Young Generation Network při Evropské nukleární společnosti a~International Youth Nuclear Congress~(IYNC). Sdružuje mladé lidi do 36 let z~jaderného odvětví. Hlavním cílem Mladé generace je zajistit setkávání členů, a~tak pomoci jejich rozvoji, získávání kontaktů a~sociálních vazeb v~jaderné komunitě. Každoročně pořádá pro členy CYG několik akcí jako jsou například exkurze do tuzemských i~zahraničních jaderných provozů (Aprílové setkání) či pravidelná setkání se slovenskou Mladou generací.}


\newpage
\section{Oceněné studentské práce \thisYear}
 {\normalsize Na Mikulášském setkání byly vyhlášeny nejlepší bakalářské,diplomové a disertační práce v jaderných oborech za rok \thisYear. Oceněny byly následující práce:}\\

\noindent\textbf{\large{Bakalářské práce}}
\begin{table}[h]
    \begin{tabular}{r p{0.85\textwidth}}
        \oceneny{I.}{\name1}{\article1}
        \oceneny{II.}{\name2}{\article2}
        \oceneny{III.}{\name3}{\article3}
    \end{tabular}
\end{table}


\noindent\textbf{\large{Diplomové práce}}
\begin{table}[h]
    \begin{tabular}{r p{0.85\textwidth}}
        \oceneny{I.}{\nameDP1}{\articleDP1}
        \oceneny{II.}{\nameDP2}{\articleDP2}
        \oceneny{III.}{\nameDP3}{\articleDP3}
    \end{tabular}
\end{table}


% optionally add Disertations
\vfill

\newpage
\pagestyle{paper}

\phantomsection
\addcontentsline{toc}{section}{Články}

\input{clanky.tex}


\newpage
\pagestyle{patickaSponzoru}

\vspace*{13cm}

\begin{table}[h]%
    \begin{tabular}{p{0.2\textwidth} p{0.8\textwidth}}%
        \multicolumn{2}{l}{Jaderná energetika v pracích mladé generace -- \thisYear} \\
                            &                           \\
        Nakladatel:         & Česká nukleární společnost, z. s. (org. č.) \\
                            & V Holešovickách 747/2     \\
                            & 180 00 Praha              \\
        Autor:              & kolektiv autorů           \\
        Rok vydání: 	    & 2021, vydání první        \\
        Forma vydání:       & vydáno pouze na CD-ROM    \\    
        Materiály sestavil: & Matěj Rzehulka            \\
        ISBN:               & TODO                      \\
                            &                           \\
        \multicolumn{2}{l}{Příspěvky jednotlivých autorů nebyly textově ani jazykově upravovány}
    \end{tabular}%
\end{table}%
\vfill

\end{document}
