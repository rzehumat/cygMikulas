\documentclass[11pt,oneside]{article}

% proper czech language
\usepackage[utf8]{inputenc}
\usepackage[T1]{fontenc}
\usepackage[czech]{babel}

\usepackage{pdfpages}
\usepackage{fancyhdr}
\usepackage{hyperref}
\usepackage{titlesec}
\usepackage{booktabs}

\usepackage{xargs}
\usepackage[bottom=2cm,left=2cm,right=2cm,top=2cm]{geometry}

% to preserve links into pdfpages
\usepackage{pax}

\titleformat*{\section}{\LARGE\bfseries}

% hide horizontal lines, cause they are already in the articles
\renewcommand{\headrulewidth}{0pt}
\renewcommand{\footrulewidth}{0pt}

\newcommand{\clanek}[3]{
    \includepdf[
        pages=-,
        pagecommand={},
        addtotoc={
            1,subsection,2,
            #3 (\textit{#2}),
            #1}
    ]
    {./papers/#1.pdf}
}

% sponsors' footer for title page
\newcommandx{\sponzori}[5]{%
       \textit{Za podpory}
       \vspace{0.1cm}
       \hrule
\begin{figure}[b]
         \href{https://www.fekt.vut.cz/}{\includegraphics[height=45px]{img/sponzori/#1}}
         \qquad
         \href{http://www.spolky-csvts.cz/cns/cyg/}{\includegraphics[height=45px]{img/sponzori/#2}}
         \qquad
         \href{http://www.spolky-csvts.cz/cns/}{\includegraphics[height=45px]{img/sponzori/#3}}
         \qquad
         \href{https://www.ujv.cz/}{\includegraphics[height=45px]{img/sponzori/#4}}
         \qquad
         \href{http://www.csvts.cz/index.php/cs/}{\includegraphics[height=45px]{img/sponzori/#5}}
\end{figure}
}

\fancypagestyle{patickaSponzoru}
{
    \fancyhead{}
    \fancyfoot[C]{
        \href{https://www.fekt.vut.cz/}{\includegraphics[height=35px]{./img/sponzori/fekt.png}}
        \qquad
        \href{http://www.spolky-csvts.cz/cns/cyg/}{\includegraphics[height=35px]{./img/sponzori/cyg.png}}
        \qquad
        \href{http://www.spolky-csvts.cz/cns/}{\includegraphics[height=35px]{./img/sponzori/cns.png}}
        \qquad
        \href{https://www.ujv.cz/}{\includegraphics[height=35px]{./img/sponzori/ujv.png}}
        \qquad
        \href{http://www.csvts.cz/index.php/cs/}{\includegraphics[height=35px]{./img/sponzori/csvts.png}}
    }
}

\fancypagestyle{paper}
{
    \fancyhead{}
    \fancyfoot[C]{\thepage}
}

\newcommand{\event}[4]{
    #1 & -- & #2 & #4 \\
     &  &  & (\textit{#3}) \\
}

\newcommand{\bigEvent}[2]{
    \multicolumn{3}{l}{\Large{\textbf{#1}}}  & \Large{\textbf{#2}} \\
}

\newcommand{\mezera}{
    \rule{0pt}{1ex} & & & \\
}

\newcommand{\oceneny}[3]{
    #1~místo & \textit{#2} #3 \\
}

\title{to mikulas}
\author{rzehumat}

\date{April 2021}

\pagestyle{fancy}
\fancyhead{}
\fancyfoot[C]{\thepage}

\begin{document}

\begin{titlepage}
   \begin{center}
       \vspace*{4cm}

       \LARGE{\textbf{Jaderná energetika v~pracích mladé generace~--~2020}}

       \vspace{0.8cm}

       \Large{Mikulášské setkání Mladé generace ČNS}

       \vfill
            
       Online\\
       27.~1.~2021\\
       
       \vfill
       
       \sponzori{fekt.png}{cyg.png}{cns.png}{ujv.png}{csvts.png}
            
   \end{center}
\end{titlepage}


\tableofcontents
\newpage

\setcounter{secnumdepth}{-1}

\pagestyle{patickaSponzoru}
\section{Program setkání}
\Large{Středa 27.~1.~2021}
\begin{table}[h]
    \begin{tabular}{r c l p{0.85\textwidth}}
        \bigEvent{9:00}{Oficiální zahájení setkání}
        9:00 & -- & 9:10 & Úvodní slovo garanta setkání \\
         \mezera
        \bigEvent{9:10}{Prezentace oceněných prací studentů}
        \event{9:10}{9:30}{Matyáš Junek}{Energetické využití jaderné fúze}
        \event{9:30}{9:50}{Tomáš Křinecký}{Prodlužování životnosti JE Dukovany}
        \event{9:50}{10:10}{Robin Krempaský}{Rekonstrukce výkonu AZ metodou RBF pro monitorování reaktorů VVER}
        \event{10:10}{10:30}{Miroslav Vejvoda}{Modelování vyhořívajících absorbátorů ve výpočtech neutronově-fyzikálních charakteristik aktivních zón}
        \event{10:30}{10:45}{Filip Fejt}{Analýza termohydraulických modelů reaktoru VR-1 s využitím trojrozměrné kinetiky}
        \mezera
        \bigEvent{11:00}{Prezentace mladých odborníků}
        \event{11:00}{11:15}{Jan Syblík}{Subkanálová analýza VVER-440 pomocí kódu SUBCHANFLOW}
        \event{11:15}{11:30}{Bence Mészáros}{Výzkum materiálových parametrů koria pomocí tavení ve studeném kelímku}
        \event{11:30}{11:45}{Michal Cihlář}{Vliv doby expozice konstrukčních materiálů v tavenině NaF-NaBF4 při statických korozních testech}
        \event{12:00}{12:15}{Ondřej Pašta}{Testování debris fretting jevu v CVŘ}
        \event{12:15}{12:30}{Zbyněk Hlaváč}{Ultrazvuková kontrola betonu biologického stínění  jaderné elektrárny Greifswald s použitím jádrových vývrtů}
        \event{12:30}{12:45}{Martin Ševeček}{Proč zavádět nové typy jaderných paliv a jaká jsou jejich omezení?}
        \mezera
        \bigEvent{12:45}{Ukončení Mikulášského setkání}
    \end{tabular}
\end{table}  

\newpage

\pagestyle{patickaSponzoru}

\section{O Mikulášském setkání}
{\normalsize Mikulášské setkání je akcí Mladé generace České nukleární společnosti určené především mladým lidem studujícím či pracujícím v~jaderné oblasti. Smyslem setkání je propojit mladé lidi z~různých koutů České republiky a~umožnit jim prezentovat jejich práci. Hlavní náplní Mikulášského setkání je prezentace oceněných bakalářských a~magisterských prací v~soutěži, která je pravidelně pořádána Českou nukleární společností společně s~ÚJV Řež.}

\vspace{1cm}

\section{O Mladé generaci ČNS}
{\normalsize Mladá generace České nukleární společnosti (CYG - Czech Young Generation) byla založena 7.~7.~1997. CYG je jednou ze sekcí České nukleární společnsti a~je také členem Young Generation Network při Evropské nukleární společnosti a~International Youth Nuclear Congress~(IYNC). Sdružuje mladé lidi do 36 let z~jaderného odvětví. Hlavním cílem Mladé generace je zajistit setkávání členů, a~tak pomoci jejich rozvoji, získávání kontaktů a~sociálních vazeb v~jaderné komunitě. Každoročně pořádá pro členy CYG několik akcí jako jsou například exkurze do tuzemských i~zahraničních jaderných provozů (Aprílové setkání) či pravidelná setkání se slovenskou Mladou generací.}


\newpage
\section{Oceněné studentské práce 2020}
{\normalsize Na Mikulášském setkání byly vyhlášeny nejlepší bakalářské,diplomové a disertační práce v jaderných oborech za rok 2020. Oceněny byly následující práce:}\\

\noindent\textbf{\large{Bakalářské práce}}
\begin{table}[h]
    \begin{tabular}{r p{0.85\textwidth}}
        % TODO fix these ugly ":" -- if not empty, than :
        \oceneny{I.}{Miroslav Vejvoda:}{Modelování vyhořívajících absorbátorů ve výpočtech neutronově-fyzikálních charakteristik aktivních zón}
        \oceneny{II.}{Robin Krempaský:}{Rekonstrukce výkonu AZ metodou RBF pro monitorování reaktorů VVER}
        \oceneny{III.}{Matyáš Junek:}{Energetické využití jaderné fúze}
        \oceneny{III.}{Tomáš Křinecký:}{Prodlužování životnosti JE Dukovany}
    \end{tabular}
\end{table}  


\noindent\textbf{\large{Diplomové práce}}
\begin{table}[h]
    \begin{tabular}{r p{0.85\textwidth}}
        \oceneny{I.}{Ondřej Hlinka:}{Měření hydrodynamiky přestupu tepla pro vnitřní úlohu zaplavování válcové geometrie}
        \oceneny{II.}{Jana Matoušková:}{Metoda Monte Carlo pro stanovení neurčitosti koeficientu násobení vlivem neurčitosti izotopického složení vyhořelého jaderného paliva}
        \oceneny{III.}{Michal Cihlář:}{Vliv doby expozice konstrukčních materiálů v tavenině NaF-NaBF4 při statických korozních testech}
    \end{tabular}
\end{table}  


\noindent\textbf{\large{Dizertační práce}}
\begin{table}[h]
    \begin{tabular}{r p{0.85\textwidth}}
        \oceneny{I.}{-}{}
        \oceneny{II.}{Filip Fejt:}{Analýza termohydraulických modelů reaktoru VR-1 s využitím trojrozměrné kinetiky}
        \oceneny{III.}{-}{}
    \end{tabular}
\end{table}  
\vfill

\newpage
\pagestyle{paper}

\phantomsection
\addcontentsline{toc}{section}{Články}

\clanek{junek}{Matyáš Junek}{Energetické využití jaderné fúze}
\clanek{krinecky}{Tomáš Křinecký}{Prodlužování životnosti JE Dukovany}
\clanek{krempasky}{Robin Krempaský}{Rekonstrukce výkonu AZ metodou RBF pro monitorování reaktorů VVER}
\clanek{vejvoda}{Miroslav Vejvoda}{Modelování vyhořívajících absorbátorů ve výpočtech neutronově-fyzikálních charakteristik aktivních zón}
\clanek{syblik}{Jan Syblík}{Subkanálová analýza VVER-440 pomocí kódu SUBCHANFLOW}
\clanek{meszaros}{Bence Mészáros}{Výzkum materiálových parametrů koria pomocí tavení ve studeném kelímku}
\clanek{cihlar}{Michal Cihlář}{Vliv doby expozice konstrukčních materiálů v tavenině NaF-NaBF4 při statických korozních testech}
\clanek{pasta}{Ondřej Pašta}{Testování debris fretting jevu v CVŘ}
\clanek{hlavac}{Zbyněk Hlaváč}{Ultrazvuková kontrola betonu biologického stínění jaderné elektrárny Greifswald s použitím jádrových vývrtů}
\clanek{matouskova}{Jana Matoušková}{Metoda Monte Carlo pro stanovení neurčitosti koeficientu násobení vlivem neurčitosti izotopického složení vyhořelého jaderného paliva}


\newpage
\pagestyle{patickaSponzoru}

\vspace*{13cm}

\begin{table}[h]%
    \begin{tabular}{p{0.2\textwidth} p{0.8\textwidth}}%
        \multicolumn{2}{l}{Jaderná energetika v pracích mladé generace – 2020} \\
                            &                           \\
        Nakladatel:         & Česká nukleární společnost, z. s. (org. č.) \\
                            & V Holešovickách 747/2     \\
                            & 180 00 Praha              \\
        Autor:              & kolektiv autorů           \\
        Rok vydání: 	    & 2021, vydání první        \\
        Forma vydání:       & vydáno pouze na CD-ROM    \\    
        Materiály sestavil: & Matěj Rzehulka            \\
        ISBN:               & 978-80-02-02935-9         \\
                            &                           \\
        \multicolumn{2}{l}{Příspěvky jednotlivých autorů nebyly textově ani jazykově upravovány}
    \end{tabular}%
\end{table}%
\vfill

\end{document}
