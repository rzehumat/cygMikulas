% "cz" pro češtinu, "sk" pro slovenčinu
% \rok{2019}
% \poradi{19}
% \datum{04}{12}{06}{12}{2019}
% \clanek{Jára Cimrman}{Velmi dlouhý název mého super článku.}
% 
% 
% \autor{Jára Cimrman}{Jan Amos Komenský}
% \email{jara@liptakov.cz}{}
% \pracoviste{KJR FJFI ČVUT v Praze, V Holešovičkách 2, Praha, 180~00 ČR}{Lešno}
% 
% \abstrakt{
% 	Tento článek by si měl přečíst prostě každý. 
% }
% 
% \klicovaSlova{měření dilatačních spár, emigrace, psaní divadelních her}

\documentclass[11pt]{article}
\usepackage[czech]{babel}
\usepackage{lmodern}
\usepackage[T1]{fontenc}
\usepackage[utf8]{inputenc}
\usepackage[left=1.5cm,
		right=1.5cm,
		top=2cm,
		bottom=2cm
		]{geometry}
\usepackage{paralist}
\usepackage{marvosym}
\usepackage{float}
\usepackage{graphicx}


\renewcommand{\pic}[3]{%
\begin{figure}[H]
\centering
\includegraphics[width=0.8\textwidth]{./img/#1}
\caption{#2}
\label{#3}
\end{figure}
}

\newcommand{\at}{\MVAt}


\renewcommand{\dpic}[6]{%
\begin{figure}[H]
    \centering
    \begin{minipage}{0.49\textwidth}
        \centering
        \includegraphics[width=0.98\textwidth]{img/#1} % first figure itself
        \caption{#2}
        \label{#3}
    \end{minipage}\hfill
    \begin{minipage}{0.49\textwidth}
        \centering
        \includegraphics[width=0.98\textwidth]{img/#4} % second figure itself
        \caption{#5}
        \label{#6}
    \end{minipage}
\end{figure}
}

% \addbibresource{ref.bib}
% \usepackage[slovak,czech]{babel}
% \usepackage[margin=1.5cm]{geometry}
% \usepackage[T1]{fontenc}
% \usepackage[utf8]{inputenc}
% \usepackage{lmodern}
% \usepackage[autostyle,czech=guillemets]{csquotes}
% \documentclass{cygclanek}

% \documentclass[class=article,crop=false]{standalone}
% \usepackage[subpreambles=true]{standalone}
% \usepackage{import}

% 
% packages
%

% \newcommand{\cz}{
% 	\usepackage[czech]{babel}
% }
% 
% \newcommand{\sk}{
% 	\usepackage[slovak]{babel}
% }

% \newcommand{\jazyk}[1]{
% 	\usepackage[#1]{babel}
% }

% \usepackage[left=1.5cm,
% 		right=1.5cm,
% 		top=2cm,
% 		bottom=2cm
% 		]{geometry}
% \usepackage[T1]{fontenc}
% \usepackage[utf8]{inputenc}
% \usepackage{lmodern}
% \usepackage{fancyhdr}
% \usepackage{authblk}
% \usepackage{etoolbox}
% \usepackage[style=iso-numeric, backend=biber, language=czech]{biblatex}
% \usepackage{csquotes}
% \usepackage{amsthm}
% \usepackage[hidelinks,unicode]{hyperref}
% \usepackage{paralist}
% \usepackage{marvosym}
% \usepackage{float}
% \usepackage{graphicx}
% \usepackage[font=small,labelfont=bf]{caption}
% \usepackage{forloop}

% options -- styling etc
% \pagestyle{fancy}
% \fancyfoot[C]{\thepage}
% \fancyhead[C]{}
% \fancyhead[L]{\title}
% \fancyhead[R]{Mikulášské setkání 2020}

% commands -- redefinitions
% \renewcommand{\maketitle}{%
% 	\begin{center}
% 		\vspace{4mm}
% 	{\Huge \textbf{\title}} \\
% 	\vspace{7mm}
% 	{\Large \author } \\
% 
% 		% \newcounter{aa}
% 		% \forloop[1]{aa}{1}{\pocetautoru}{
% 		% 	{\Large \autor\arabic{aa}\refei{,}{\arabic{aa}}}
% 		% 	}
% 		% \\
% 	\vspace{2mm}
% 		\email
% 	\vspace{2mm}
% 	\begin{compactenum}
% 		\centering
% 		\begin{scriptsize}
% 		\item \label{vut} VUT v Brně, Fakulta elektrotechniky a komunikačních technologií, Ústav elektroenergetiky, Technická 3058/10, 616 00 Brno, Česká republika
% 		\item \label{ujv} UJV
% 		\item \label{ujf} UJF
% 			% \newcounter{ac}
% 			% \forloop[1]{ac}{1}{\pocetautoru}{
% 			% \item \label{\arabic{ac}} \instituce\arabic{ac}
% 			% }
% 		\end{scriptsize}
% 	\end{compactenum}
% 	\end{center}
% }

% commands -- new
% \newcommand{\footnotei}[2]{%
% \mbox{%
% \setbox0\hbox{#1}%
% \copy0%
% \hspace{-\wd0}}%
% \footnote{#2}%
% }
% 
% \newcommand{\refei}[2]{%
% \mbox{%
% \setbox0\hbox{#1}%
% \copy0%
% \hspace{-\wd0}}%
% $^{\ref{#2}}$%
% }
% 
% \newcommand{\pic}[3]{%
% \begin{figure}[H]
% \centering
% \includegraphics[width=0.8\textwidth]{./img/#1}
% \caption{#2}
% \label{#3}
% \end{figure}
% }
% 
% \newcommand{\at}{\MVAt}
% 
% 
% \newcommand{\dpic}[6]{%
% \begin{figure}[H]
%     \centering
%     \begin{minipage}{0.49\textwidth}
%         \centering
%         \includegraphics[width=0.98\textwidth]{img/#1} % first figure itself
%         \caption{#2}
%         \label{#3}
%     \end{minipage}\hfill
%     \begin{minipage}{0.49\textwidth}
%         \centering
%         \includegraphics[width=0.98\textwidth]{img/#4} % second figure itself
%         \caption{#5}
%         \label{#6}
%     \end{minipage}
% \end{figure}
% }
% 
% 
% \usepackage[czech]{babel}
% % \usepackage[slovak]{babel}
% % \usepackage[slovak,czech]{babel}
\usepackage[left=1.5cm,right=1.5cm,top=2cm,bottom=2cm]{geometry}
\usepackage[T1]{fontenc}
\usepackage[utf8]{inputenc}
\usepackage{lmodern}
\usepackage{fancyhdr}
\usepackage{authblk}
\usepackage{etoolbox}

% options -- styling etc
\pagestyle{fancy}
\fancyfoot[C]{\thepage}
\fancyhead[C]{Mikulas TODO}
\fancyhead[R]{}
\fancyhead[L]{}

% commands -- redefinitions
% \renewcommand{\maketitle}{%
% 	\begin{center}
% 	{\huge \textbf{\@title}} \\
% 	{\large \@author}
% 	\end{center}
% }

% \addbibresource{ref.bib}
\begin{document}

\title{Velmi, ale opravdu velmi dlouhý název članku}

% \author[1]{Cimrman}
% \author[2]{Proxy}
% \affil[1]{Department of Mathematics, University X}
% \affil[2]{Department of Biology, University Y}
% \author{Jára Cimrman, Zdeněk Svěrák, Petr Pavel}
% \author{Jára Cimrman\refei{,}{vut} Zdeněk Svěrák\refei{,}{ujv} Petr Pavel\refei{}{ujf}}
% \author{bb}
% \newcounter{pocetautoru}
% \setcounter{pocetautoru}{3}
% \newcommand{\autor1{Cimrman}}
% \newcommand{\instituce1{ORF}}
% \newcommand{\autor2{Bruncvík}}
% \newcommand{\instituce2{VUT}}
% \newcommand{\autor3{Praotec Čech}}
% \newcommand{\instituce3{VŠB}}

% \providecommand{\keywords}[1]{\textbf{\textit{Klíčová slova:}} #1}
% \newcommand{\email}{cimrman\at vut.cz}
% \instituce{(UJV,FJFI,VUT)}
% \maily{(cimrman\@ujv.cz,sverak\@fjfi.cz,pavel\@vut.cz)}

\author{bb}
\maketitle
\begin{abstract}
	
	Cimrman se však nevzdává a ze svého liptákovského ústraní zasahuje konvenční pohádkovou tvorbu jedovatými šípy svých kritických výpadů: "Kdo kdy potkal vlka, který mluví!" To Cimrman nemilosrdně buší do Červené Karkulky. A pokračuje: "Které zvíře dokáže sníst v celku tak veliká sousta, jako jsou babička, Karkulka a třená bábovka? Která dusí! Učíme děti ve školách o zažívacích procesech. Vykládáme jim, jak se potrava rozmělněná v ústech mísí se slinami, jak je dále zpracovávána žaludečními šťávami a peristaltikou střev. Vím, myslivec přišel poměrně brzo, takže trávení teprve započalo, ale přesto nenajdete dítě, které by uvěřilo, že babička s Karkulkou vyšly z vlkových útrob v nažehlených šatečkách a škrobeném neposlintaném fěrtošku."
	ctete to
\end{abstract}
% \keywords{foo, bar}

\section{Úvod}
Cimrman % \cite{pelikan} se \at 
však \cite{einstein} nevzdává a ze svého liptákovského ústraní zasahuje konvenční pohádkovou tvorbu jedovatými šípy svých kritických výpadů: "Kdo kdy potkal vlka, který mluví!" To Cimrman nemilosrdně buší do Červené Karkulky. A pokračuje: "Které zvíře dokáže sníst v celku tak veliká sousta, jako jsou babička, Karkulka a třená bábovka? Která dusí! Učíme děti ve školách o zažívacích procesech. Vykládáme jim, jak se potrava rozmělněná v ústech mísí se slinami, jak je dále zpracovávána žaludečními šťávami a peristaltikou střev. Vím, myslivec přišel poměrně brzo, takže trávení teprve započalo, ale přesto nenajdete dítě, které by uvěřilo, že babička s Karkulkou vyšly z vlkových útrob v nažehlených šatečkách a škrobeném neposlintaném fěrtošku."
\section{Teorie}
Ke druhé změně nás vedla Cimrmanova ručně psaná poznámka na titulním listě hry: "Nedělat přestávku, jinak utečou." My tomuto nebezpečí čelíme tím, že přestávku sice děláme, ale zařazujeme ji hned za třetí obraz hry, což je tak nečekaně brzy, že se pohádka ani nestačí rozjet. Podle odhadu našeho psychologa dr. Pšeničky se publikum o přestávce rozdělí na dva tábory. Jedni by rádi odešli domů, ale bude jim prý líto, že vynaložili tolik peněz na tak krátký čas zábavy. Druzí by také rádi odešli domů, ale ti zase setrvávají ze zvědavosti, zda bude druhá část představení stejně slabá jako první. A kromě toho zamykáme hlavní dveře.


\subsection{Notová osnova}
Ostatně divák, který by si nechal ujít druhou půli večera, by se ošidil o výstup v dějinách inscenační tvorby zcela ojedinělý. Jedná se o proměnu jedné osoby v osobu jinou, která se odehraje přímo před očima diváků, a to podle vlastního Cimrmanova vynálezu. Tento výjev vzbudil ve své době světový rozruch, především na Litoměřicku, i byl označován jako "zázrak divadelní techniky."

\pic{both.png}{Nejaky obrazek.}{fig:takakokotina}

\section{Experiment}
Rád bych teď využil té skutečnosti % \ref{fig:takakokotina}, 
že má dnes službu jevištní mistr, který vynález podle Cimrmanova původního nákresu rekonstruoval, takže by nám o něm mohl říci několik zajímavostí. 
\dpic{both.png}{Jeden obrazek.}{fig:prvni}{calibration.png}{Druhý obr.}{fig:druhy}

\subsection{Veselý železničář}
(Zavolá do opony a podrží ji rozevřenou. Nikdo se však neobjeví, a tak přednášející zajde za~% \ref{fig:prvni} oponu a~\ref{fig:druhy} 
po chvíli přivede neochotně se tvářícího mistra.)

Pane kolego, já jsem tu hovořil o tom Cimrmanově vynálezu, a vy jste ho vlastně rekonstruoval. Buďte tak laskav a povězte divákům, jak to celé funguje. (Mistr mlčí.)

Rozumíte, já po vás nechci žádnou přednášku, jenom ten základní princip a jednu dvě zajímavosti. (Jevištní mistr mlčí.)
\section{Závěr}
Že vás ještě přerušuji: já jsem si všiml, že tam je taková soustava vodičů vzájemně propojených, že, která je přesně vyvážená, a celé je to, myslím, pevně fixováno v portále, ne?

Jevištní mistr: Žádný vodiče tam nejsou. Přednášející: Aha, tak já do toho tak nevidím. Dobře, že vás tu máme. My jenom vidíme, že jak ona tam princezna Zlatovláska stojí, tak se při plném světle uprostřed jeviště promění. Je to tak, nebo ne? (Jevištní mistr přikývne.)
\section*{Poděkování}
Přednášející: A já jsem si právě myslel, že to je způsobeno těmi vodiči, respektive jejich napětím, že se její staré rysy odstraní a nahradí novými. A to vy ovládáte u toho řídicího panelu, viďte? Jevištní mistr: Tam sedí Maurenc.



%\printbibliography[title={Literatura}]
\bibliographystyle{unsrt}
\bibliography{ref}
% \literatura{
% @article{baker,
% title = "How to bake cookies",
% journal = "Annals of baking",
% volume = "21",
% number = "6",
% pages = "325 - 336",
% year = "1994",
% issn = "0308-4844",
% doi = "https://doi.org/11.1216/0306-4549(94)95528-0",
% url = "http://www.sciencedirect.com/science/article/pii/0396458994900280",
% author = "T. Baker and M. Cook",
% }
% 
% @online{wnn,
%   journal = {World Nuclear News},
%   title = {Decommissioning of UK research reactor approved},
%   publisher = {World Nuclear Association},
%   address = {London},
%   year = {2020},
%   urldate = {2020-08-09},
%   url = {https://www.world-nuclear-news.org/WR-Decommissioning-of-UK-research-reactor-approved-1808154.html},
% }
% 
% }
% \napisPrispevek
\end{document}
