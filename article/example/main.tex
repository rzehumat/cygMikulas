\documentclass{cygclanek}
\addbibresource{ref.bib}
\begin{document}

\title{Velmi, velmi dlouhý název článku}

\names{
    \name[*]{Jára Cimrman}{vut}, \name{Zdeněk Svěrák}{ujv}, \name{Petr Pavel}{ujf}
}
\institutions{
    \institution{vut}{Vysoké učení technické}
    \institution{ujv}{ÚJV Řež}
    \institution{ujf}{ÚJF AVČR}
}

\email{cimrman@vut.cz}
\rok{2035}

\maketitle
\begin{abstract}
	Cimrman se však nevzdává a ze svého liptákovského ústraní zasahuje konvenční
  pohádkovou tvorbu jedovatými šípy svých kritických výpadů: \uv{Kdo kdy potkal
  vlka, který mluví!} To Cimrman nemilosrdně buší do Červené Karkulky. A
  pokračuje: \uv{Které zvíře dokáže sníst v celku tak veliká sousta, jako jsou babička, Karkulka a třená bábovka? Která dusí! Učíme děti ve školách o zažívacích procesech. Vykládáme jim, jak se potrava rozmělněná v ústech mísí se slinami, jak je dále zpracovávána žaludečními šťávami a peristaltikou střev. Vím, myslivec přišel poměrně brzo, takže trávení teprve započalo, ale přesto nenajdete dítě, které by uvěřilo, že babička s Karkulkou vyšly z vlkových útrob v nažehlených šatečkách a škrobeném neposlintaném fěrtošku.}
\end{abstract}
\keywords{foo, bar, spam, eggs}

\section{Úvod}

Cimrman \cite{pelikan} se však nevzdává a ze svého liptákovského ústraní
zasahuje konvenční pohádkovou tvorbu jedovatými šípy svých kritických výpadů:
\uv{Kdo kdy potkal vlka, který mluví!} To Cimrman nemilosrdně buší do Červené
Karkulky. A pokračuje: \uv{Které zvíře dokáže sníst v celku tak veliká sousta, jako jsou babička, Karkulka a třená bábovka? Která dusí! Učíme děti ve školách o zažívacích procesech. Vykládáme jim, jak se potrava rozmělněná v ústech mísí se slinami, jak je dále zpracovávána žaludečními šťávami a peristaltikou střev. Vím, myslivec přišel poměrně brzo, takže trávení teprve započalo, ale přesto nenajdete dítě, které by uvěřilo, že babička s Karkulkou vyšly z vlkových útrob v nažehlených šatečkách a škrobeném neposlintaném fěrtošku.}

\section{Teorie}
Ke druhé změně nás vedla Cimrmanova ručně psaná poznámka na titulním listě hry:
\uv{Nedělat přestávku, jinak utečou.} My tomuto nebezpečí čelíme tím, že
přestávku sice děláme, ale zařazujeme ji hned za třetí obraz hry, což je tak
nečekaně brzy, že se pohádka ani nestačí rozjet. 
\begin{equation}
  D\nabla^2\phi - \Sigma_a\phi + \nu\Sigma_f\phi = \frac{1}{v}\frac{\partial
  \phi}{\partial t}
  \label{difuzka}
\end{equation}

Zde se ozkážeme na rovnici \eqref{difuzka}. 

Podle odhadu našeho psychologa
dr. Pšeničky se publikum o přestávce rozdělí na dva tábory. Jedni by rádi
odešli domů, ale bude jim prý líto, že vynaložili tolik peněz na tak krátký čas
zábavy. Druzí by také rádi odešli domů, ale ti zase setrvávají ze zvědavosti,
zda bude druhá část představení stejně slabá jako první. A kromě toho zamykáme
hlavní dveře.


\subsection{Notová osnova}
Ostatně divák, který by si nechal ujít druhou půli večera, by se ošidil o
výstup v dějinách inscenační tvorby zcela ojedinělý. Jedná se o proměnu jedné
osoby v osobu jinou, která se odehraje přímo před očima diváků, a to podle
vlastního Cimrmanova vynálezu. Tento výjev vzbudil ve své době světový rozruch,
především na Litoměřicku, i byl označován jako \uv{zázrak divadelní techniky.}

\obr{fjfi.pdf}{Nejaky obrazek bez nepovinneho \uv{parametru}. Vypada trochu moc
velky.}{fig:moc-velky-obrazek}

\section{Experiment}
Rád bych teď využil té skutečnosti \ref{fig:moc-velky-obrazek}, že má dnes
službu jevištní mistr, který vynález podle Cimrmanova původního nákresu
rekonstruoval, takže by nám o~něm mohl říci několik zajímavostí, zejména
ověření, že 
\begin{align}
  a &= b \label{rovnice_a} \,,\\
  b &= c \label{rovnice_b} \,,
\end{align}

skutečně platí. Na každou z těch rovnic se můžu odkázat -- třeba takto
\eqref{rovnice_a} a takto \eqref{rovnice_b}.

\dobr{fjfi.pdf}{Jeden obrazek. Zřejmě by bylo dobré udělat je stejně velké.
Proto v obr.~\ref{fig:srovnany} nastavíme velikost pomocí nepovinneho
parametru.}{fig:prvni}{symbol_cvut_konturova_verze_cb.pdf}{Druhý obr.}{fig:druhy}


\dobr{fjfi.pdf}{Jeden obrazek. Pomocí nepovinných parametrů byla nastavena
šířka.}{fig:srovnany}{symbol_cvut_konturova_verze_cb.pdf}{Druhý
obr.}{fig:jiny}[0.7][0.98]

\begin{table}[H]
\centering
\begin{tabular}{ccc}
	\toprule
$\rho$ [\textcent] & $T_e$ [s] & $T_d$ [s] \\
\midrule
3,6 & $(312,95 \pm 0,01)$ & $(216,92 \pm 0,01)$ \\
6,5 & $(162,22 \pm 0,02)$ & $(112,44 \pm 0,01)$ \\
9,8 & $(96,79 \pm 0,09)$ & $(67,61 \pm 0,06)$ \\
12,8 & $(68,06 \pm 0,02)$ & $(47,18 \pm 0,01)$ \\
15,5 & $(51,36 \pm 0,06)$ & $(35,60 \pm 0,04)$ \\
19,0 & $(38,11 \pm 0,04)$ & $(26,47 \pm 0,03)$ \\
\bottomrule
\end{tabular}
\caption{Tabulka -- návrh tzv. \uv{čistá}.}
\label{mer}
\end{table}


\subsection{Veselý železničář}
(Zavolá do opony a podrží ji rozevřenou. \uv{Nikdo} se však neobjeví, a tak přednášející zajde za~\ref{fig:prvni} oponu a~\ref{fig:druhy} po chvíli přivede neochotně se tvářícího mistra.)

\graf{both.png}{Nejaky graf.}{fig:takygraf}

Pane kolego, já jsem tu hovořil o tom Cimrmanově vynálezu, a vy jste ho vlastně rekonstruoval. Buďte tak laskav a povězte divákům, jak to celé funguje. (Mistr mlčí.)

\dgraf{both.png}{Jeden graf.}{fig:dalsi}{calibration.png}{Druhý graf.}{fig:ddalsi}
Rozumíte, já po vás nechci žádnou přednášku, jenom ten základní princip a jednu dvě zajímavosti. (Jevištní mistr mlčí.)

Jevištní mistr: Žádný vodiče tam nejsou. Přednášející: Aha, tak já do toho tak nevidím. Dobře, že vás tu máme. My jenom vidíme, že jak ona tam princezna Zlatovláska stojí, tak se při plném světle uprostřed jeviště promění. Je to tak, nebo ne? (Jevištní mistr přikývne.)

\begin{table}[H]
\centering
\begin{tabular}{|c|c|c|}
\hline
$\rho$ [\textcent] & $T_e$ [s] & $T_d$ [s] \\
\hline
3,6 & $(312,95 \pm 0,01)$ & $(216,92 \pm 0,01)$ \\
\hline
6,5 & $(162,22 \pm 0,02)$ & $(112,44 \pm 0,01)$ \\
\hline
9,8 & $(96,749 \pm 0,009)$ & $(67,061 \pm 0,006)$ \\
\hline
12,8 & $(68,06 \pm 0,02)$ & $(47,18 \pm 0,01)$ \\
\hline
15,5 & $(51,36 \pm 0,06)$ & $(35,60 \pm 0,04)$ \\
\hline
19,0 & $(38,111 \pm 0,004)$ & $(26,417 \pm 0,003)$ \\
\hline
\end{tabular}
\caption{Tabulka \uv{plná}.}
\label{ver}
\end{table}


\section{Závěr}
Že vás ještě přerušuji: já jsem si všiml, že tam je taková soustava vodičů vzájemně propojených, že, která je přesně vyvážená, a celé je to, myslím, pevně fixováno v portále, ne?
\obr{fjfi.pdf}{Nejaky obrazek s nepovinnym parametrem sirka = 0.1 strany.}{fig:takapi}[0.1]

Jevištní mistr: Žádný vodiče tam nejsou. Přednášející: Aha, tak já do toho tak nevidím. Dobře, že vás tu máme. My jenom vidíme, že jak ona tam princezna Zlatovláska stojí, tak se při plném světle uprostřed jeviště promění. Je to tak, nebo ne? (Jevištní mistr přikývne.)
\section*{Poděkování}
Přednášející: A já jsem si právě myslel, že to je způsobeno těmi vodiči, respektive jejich napětím, že se její staré rysy odstraní a nahradí novými. A to vy ovládáte u toho řídicího panelu, viďte? Jevištní mistr: Tam sedí Maurenc.



\printbibliography[title={Literatura}]

\end{document}
